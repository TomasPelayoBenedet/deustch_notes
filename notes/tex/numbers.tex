\section{Numbers (Die Zahlen)}\label{sec:numbers}

In this section, we will explore how numbers work in German, looking at their distinctive patterns and rules. Understanding the German number system is a key step in mastering this language, as it differs from English in several interesting ways.

In german number 0 is \textit{null} and from 1 to 20 are writen in Table \ref{tab:numbers1.20}. For the rest of the numbers we can follow specific patterns to create the name of the numbers. These will be present in the following paragraphs.

\begin{table}[H]
    \centering
    \begin{tabular}{|c|c|c|c|c|c|c|c|c|c|}
    \hline
    1 & 2 & 3 & 4 & 5 & 6 & 7 & 8 & 9 & 10 \\
    eins & zwei & drei & vier & fünf & sechs & sieben & acht & neun & zehn \\ \hline
    11 & 12 & 13 & 14 & 15 & 16 & 17 & 18 & 19 & 20 \\
    elf & zwölf & dreizehn & vierzehn & fünfzehn & sechzehn & siebzehn & achtzehn & neunzehn & zwanzig \\ \hline
    \end{tabular}
    \caption{Numbers from 1 to 20 and their names in German}
    \label{tab:numbers1.20}
\end{table}

\TODO{Explain complex numbers}