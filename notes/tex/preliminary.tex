\section{Preliminary}\label{sec:preliminary}

This section establishes the basic concepts and terminology needed to understand the rest of these German language notes. These fundamentals will serve as a foundation and will be referenced frequently throughout the document, so it's essential to familiarize yourself with them before proceeding.

Nouns must always begin with a capital letter, and in certain cases, when letters appear in this document in parentheses next to the noun, this indicates how the plural is formed. E.g.: Siete(n). If there is no parentheses don't mean that there isn't plural, means that is not writen in that example. 

\subsection{Document Structure}

Document follows:

\begin{itemize}
    \item Section \ref{sec:pronunciation}: Pronunciation. This section explains the pronunciation of the german words.
    \item Section \ref{sec:numbers}: Numbers. This section explains the construction of the numbers in german.
    \item Section \ref{sec:verbs}: Verbs: This section explains how to conjugate verbs
\end{itemize}