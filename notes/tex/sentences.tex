\section{Sentences}\label{sec:sentences}

In this section, we will dive into the structure of sentences in German, focusing on their unique word order and grammatical rules. Learning how to construct sentences is an essential step in mastering German, as its syntax differs significantly from English, offering both challenges and opportunities to develop a deeper understanding of the language.

In declarative sentences, the verb ALWAYS goes in the second position. E.g.:

\begin{center}
    Ich \underline{\textbf{gehe}} am Montag mit Pedro ins Stadium

    Am Montag \underline{\textbf{gehe}} ich mit Pedro ins Stadium
\end{center}

\subsection{Temporal Kausal Modal Lokal}\label{subsec:tkml}

This is the order to create sentences:

\begin{center}
    \textbf{T}emporal \textbf{K}ausal \textbf{M}odal \textbf{L}okal
\end{center}

e.g.:

\[
\syntactic{Ich}{Subject}{1em}
\syntactic{gehe}{Verb}{1em}
\syntactic{am Montag}{Temporal}{1em}
\syntactic{mit Pedro}{Modal}{1em}
\syntactic{ins Stadium}{Lokal}{1em}
\]

\[
\syntactic{Am Montag}{Temporal}{1em}
\syntactic{gehe}{Verb}{1em}
\syntactic{ich}{Subject}{1em}
\syntactic{mit Pedro}{Modal}{1em}
\syntactic{ins Stadium}{Lokal}{1em}
?
\]

\TODO{Review this section and look for complete it}